\documentclass[10pt,a4paper]{extarticle}
\pagenumbering{gobble}
\usepackage[left=0.6in,top=0.6in,right=0.6in,bottom=0.6in]{geometry}
\usepackage{hyperref}
\usepackage{titlesec}
\usepackage{enumitem}
\usepackage{graphicx}
\titleformat{\section}{\large\bfseries\scshape\raggedright}{}{0em}{}[\titlerule]
\titlespacing*{\section}{0pt}{12pt}{8pt}
\begin{document}
\begin{center}
    \begin{minipage}{\textwidth}
        \centering
        {\LARGE\textbf{Bengüsu Çoban}} \hspace{2pt} {\LARGE{Bilgisayar Mühendisliği Öğrencisi}}\\[10pt]
        \href{mailto:bengusucobann@gmail.com}{bengusucobann@gmail.com} \textbullet\
        +90(507)5897645 \textbullet\
        Konya, Türkiye \textbullet\
        \href{https://linkedin.com/in/bengüsu-ç-a26b6031b}{linkedin.com/in/bengüsu-ç-a26b6031b}
    \end{minipage}
\end{center}
\section{Profil}
Son sınıf bilgisayar mühendisliği öğrencisiyim ve yapay zeka ile veri analizine derin bir ilgim var. Bu alanlarda becerilerimi geliştirmek ve pratik deneyim kazanmak için fırsatlar arıyorum. Yapay zeka algoritmaları ve veri analizi teknikleri konusunda bilgi sahibiyim ve projeler aracılığıyla uygulamalı deneyim edinmeyi hedefliyorum. Amacım, veri odaklı çözümler geliştirmek ve yapay zeka teknolojilerinin potansiyelini en üst düzeye çıkarmaktır.

\section{Profesyonel Deneyim}
\textbf{Anadolu Birlik Holding - Konya Şeker Fabrikası} \hfill Konya, Türkiye\\
\textit{Ağ Stajyeri} \hfill Ağustos 2024 -- Eylül 2024
\begin{itemize}[leftmargin=*,noitemsep,topsep=0pt]
    \item BT Altyapı ve Operasyonlar departmanında stajyer olarak, şirketin bilgi teknolojileri altyapısının ve operasyonel süreçlerinin yönetimine destek sağladım.
\end{itemize}

\textbf{Meram Elektrik Dağıtım A.Ş.} \hfill Konya, Türkiye\\
\textit{İş Zekası ve Raporlama Stajyeri} \hfill Temmuz 2024 -- Ağustos 2024
\begin{itemize}[leftmargin=*,noitemsep,topsep=0pt]
    \item Power BI kullanarak çeşitli iş birimleri için etkileşimli raporlar ve görselleştirmeler oluşturmak üzere veri analizleri gerçekleştirdim.
    \item Veritabanlarını analiz ederek KPI'ları takip etmek için veri modelleri geliştirdim ve raporlama süreçlerini otomatikleştirdim.
    \item Stratejik karar verme süreçlerine veri odaklı içgörüler sağlamak için çeşitli veri kaynaklarını entegre ettim.
\end{itemize}

\textbf{Endüstriyel Elektrik Elektronik Sanayi ve Ticaret A.Ş.} \hfill Konya, Türkiye\\
\textit{Elektronik AR-GE ve Güneş Enerjisi Sistemleri Stajyeri} \hfill Ağustos 2022 -- Eylül 2022
\begin{itemize}[leftmargin=*,noitemsep,topsep=0pt]
    \item SketchUp kullanarak güneş panelleri tasarladım ve projeler için teklif hazırlama süreçlerine destek verdim.
    \item Elektronik AR-GE departmanında çalışarak, yenilikçi elektronik ürünlerin geliştirilmesine ve mevcut ürünlerin iyileştirilmesine katkıda bulundum.
\end{itemize}

\section{Eğitim}
\textbf{Konya Teknik Üniversitesi} \hfill Konya, Türkiye\\
\textit{Bilgisayar Mühendisliği} \hfill Eylül 2022 -- Haziran 2025

\textbf{Konya Teknik Üniversitesi} \hfill Konya, Türkiye\\
\textit{Elektrik - Elektronik Mühendisliği} \hfill Ekim 2020 -- Haziran 2022

\section{Yetenekler}
\textbf{Veri Analizi \& Görselleştirme:} Power BI, Pandas, Matplotlib, Seaborn, Pentaho\\
\textbf{Programlama Dilleri:} Python, SQL\\
\textbf{Diller:} İngilizce (Okuma \& Yazma: İleri, Konuşma: Orta)
\end{document}
